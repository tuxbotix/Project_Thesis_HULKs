\chapter*{Statutory Declaration}

\noindent
I, \{Your Name\}, born on \{Your Birthday\} in \{Play of Birth\}, hereby declare on oath that I compiled this master thesis, submitted to Technische Universit\"at Hamburg-Harburg (TUHH), on my own. I only used the declared sources and auxiliaries.

\vspace*{5em}

\noindent
{\rule{15em}{.5pt}}\hfill{\rule{15em}{.5pt}}\\
Place and Date \hfill Signature

\begingroup
\renewcommand{\cleardoublepage}{}
\renewcommand{\clearpage}{}
\chapter*{Eidesstattliche Erkl\"arung}
\endgroup

\noindent
Ich, \{Your Name\}, geboren am \{Your Birthday\} in \{Place of Birth\}, versichere hiermit an Eides statt,
dass ich diese von mir bei der Technischen Universit\"at Hamburg-Harburg (TUHH) vorgelegte
Masterarbeit selbstst\"andig verfasst habe. Ich habe ausschließlich
die angegebenen Quellen und Hilfsmittel benutzt.

\vspace*{5em}

\noindent
{\rule{15em}{.5pt}}\hfill{\rule{15em}{.5pt}}\\
Ort und Datum \hfill Unterschrift

%====================== ACKNOWLEDGEMENTS ======================
\newpage
\pagestyle{plain}
\addcontentsline{toc}{section}{\bfseries{ACKNOWLEDGEMENTS}}
\onecolumn % Single-column.
\if@twoside\else\raggedbottom\fi % Ragged bottom unless twoside option.
\setlength{\footskip}{8mm}
\begin{center}
	{
		\large \bf ACKNOWLEDGEMENTS\\ \vskip 1em
	}
	\vskip 1em
\end{center}
%\singlespace
%\doublespace
\hspace{8.5mm}
\vspace{-1em}
\todo{Complete this}

%====================== ABSTRACT ======================
\newpage
\pagestyle{plain}
\addcontentsline{toc}{section}{\bfseries{ABSTRACT}}
\onecolumn % Single-column.
\if@twoside\else\raggedbottom\fi % Ragged bottom unless twoside option.

\setlength{\footskip}{8mm}

\begin{center}
	{\large \bf ABSTRACT \\ \vskip 1em}
	\vskip 1em
\end{center}
%\singlespace
%\doublespace
\hspace{8.5mm}
\vspace{-1em}

Calibration of robotic joints is an important requirement for accurate operation of Robots. While the majority of previous work is in industrial setting with specialized equipment, whereas this thesis develops a process and derivation of optimal calibration poses for Joint calibration of the NAO robot by usage of its onboard cameras. The proposed method is also extensible for additional sensors. The results demonstrate that finding a set of optimal poses is crucial for obtaining acceptable results while keeping the required number of calibration poses to a minimal.
\todo{check this}


%\textbf{Keywords:} \textbf{ reaction wheel pendulum, under actuated systems, state-space modelling, PID control, LQR control}